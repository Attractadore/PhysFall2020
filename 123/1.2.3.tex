\documentclass{article}
\usepackage[english,russian]{babel}
\usepackage{amsmath}
\usepackage[parfill]{parskip}

\title{Определение моментов инерции твердый тел с помощью трифилярного подвеса}
\date{}
\author{Баргатин Михаил}

\begin{document}
    \maketitle
    \pagenumbering{gobble}
    \newpage
    \pagenumbering{arabic}

    Момент инерции для однородных тел простой формы легко вычислить математически.
    Однако для неоднородных тел или тел сложной формы это сделать не всегда возможно.
    В данном опыте предлагается измерить момент инерции сложных тел при помощи трифилярного подвеса.
    Принцип его работы следующий: к верхней неподвижной пластине на расстоянии $r$ от ее центра крепятся три
    нити, другой конец которых закреплен на расстоянии $R$ от центра нижней подвижной пластины. По периоду $T$
    крутильных колебаний нижней пластины с расположенным на ней твердым телом общей массой $m$
    можно определить их суммарный момент импульса $I$.
    
    Если в процессе малых колебания маятника затухание происходит медленно,
    можно считать, что сохраняется полная механическая энергия $E$. Ее составляющие -- 
    кинетическая энергия вращения тела с пластиной и их потенциальная энергия:
    \begin{equation}
        \frac {I \dot \phi^2} {2} + mg(h - h') = E
    \end{equation}

    В положении равновесия расстояние между пластинами $h$ и длина нити $L$ связаны следующим образом:
    \begin{equation}
        (R - r)^2 + h^2 = L^2
    \end{equation}

    Так как нить нерастяжима, при повороте на угол $\phi$ расстояние между пластинами уменьшится до $h'$:
    \begin{equation}
        (R \cos \phi - r)^2 + (R \sin \phi)^2 + h'^2 = L^2
    \end{equation}

    \begin{equation}
        (R \cos \phi - r)^2 + (R \sin \phi)^2 + h'^2 = (R - r)^2 + h^2 
    \end{equation}

    \begin{equation}
        h'^2 = h^2 - 2rR(1 - \cos \phi) \end{equation}
    Так как колебания малы воспользуемся следующим приближением:
    \begin{equation*}
        \cos \phi \approx 1 - \frac {\phi^2} {2}
    \end{equation*}

    \begin{equation}
        h'^2 = h^2 - rR\phi^2
    \end{equation}

    \begin{equation}
        h' = h \sqrt {1 - \frac {rR\phi^2} {h^2}} = h - \frac {rR\phi^2} {2h}
    \end{equation}

    Подставим результат в (1):
    \begin{equation}
        \frac {I \dot \phi^2} {2} + mg \frac {rR\phi^2} {2h} = E
    \end{equation}

    И продифференцируем по $\phi$:
    \begin{equation}
        I \dot \phi \ddot \phi + mg \frac {rR \phi \dot \phi} {h} = 0
    \end{equation}

    Получаем уравнение гармонических колебаний:
    \begin{equation}
        \ddot \phi + mg \frac {rR \phi} {Ih} = 0
    \end{equation}
    
    \begin{equation}
        \omega ^ 2 = mg \frac {rR} {Ih}
    \end{equation}
    
    Период колебаний и момент инерции тела с пластиной связаны следующим образом:
    \begin{equation*}
        T = \frac {2 \pi} {\omega}
    \end{equation*}

    \begin{equation}
        I = mg \frac {rRT^2} {4 \pi^2 h}
    \end{equation}

    Для нахождения момента инерции тела без пластины необходимо провести измерения для ненагруженной пластины.
    Так как момент инерции обладает свойством аддитивности, момент инерции самого тела будет равен разности полученных
    результатов.

\end{document}
