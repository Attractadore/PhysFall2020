\documentclass{article}
\usepackage[english,russian]{babel}
\usepackage{amsmath}
\usepackage[parfill]{parskip}

\title{Изучение колебаний струны}
\date{}
\author{Баргатин Михаил}

\begin{document}
    \maketitle
    \pagenumbering{gobble}
    \newpage
    \pagenumbering{arabic}

    \section{Поперечные волны в струне}

    Уравнение бегущей волны в струне:
    \begin{equation}
        s(x, t) = A \cos (x \frac {\omega} {u} - \omega t)
    \end{equation}
    где $A$ -- амплитуда волны, $\omega$ -- ее угловая частота, $u$ -- скорость ее распространения.
    \newline
    
    Когда волна достигает одного из концов струны, она отражается и распространяется в противоположном
    направлении. Отраженная и исходная волны интерферируют и образуют стоячую волну:
    \begin{equation}
        h(x, t) = A \cos (x \frac {\omega} {u} - \omega t) + A \cos (x \frac {\omega} {u} + \omega t + \phi)
    \end{equation}
    \begin{equation}
        h(x, t) = 2A \cos (x \frac {\omega} {u} + \frac {\phi} {2}) \cos (\omega t + \frac {\phi} {2})
    \end{equation}

    Концы струны покоятся, найдем из этих соображений частоту колебаний струны $\nu$:
    \begin{equation*}
        x = 0
    \end{equation*}
    \begin{equation}
        cos(\frac{\phi}{2}) = 0
    \end{equation}
    \begin{equation*}
        \phi = \pi
    \end{equation*}
    \begin{equation*}
        x = l
    \end{equation*}
    \begin{equation*}
        n \in N
    \end{equation*}
    \begin{equation}
        \frac {\omega} {u} l = n \pi
    \end{equation}
    \begin{equation*}
        \nu = \frac {\omega} {2 \pi}
    \end{equation*} 
    \begin{equation}
        \frac {2 \pi \nu}{u} l = n \pi
    \end{equation}
    \begin{equation}
        \nu = n \frac {u} {2 l}
    \end{equation}

    \section{Распространение колебательного процесса в струне}

    Рассмотрим вертикальные проекции сил, действующих на участок $\Delta x$ струны:
    \begin{equation}
        \rho \Delta x S \ddot h(x) = T(x + \Delta x) \sin \alpha (x + \Delta x) - T (x) \sin \alpha (x)
    \end{equation}
    где $\rho$ -- плотность материала струны, $S$ -- площадь ее поперечного сечения.
    
    Сила натяжения, действующая на участок струны $T(x)$:
    \begin{equation}
        T (x) = \sigma (x) S
    \end{equation}
    где $\sigma(x)$ -- напряжение струны.

    Будем считать, что напряжение не меняется вдоль струны:
    \begin{equation}
        \sigma (x) = const = \sigma
    \end{equation}
    
    И что $\alpha$ мал:
    \begin{equation}
        \tan \alpha (x) = h'(x)
    \end{equation}
    \begin{equation}
        \sin \alpha (x) \approx \tan \alpha (x)
    \end{equation}

    И тогда:
    \begin{equation}
        \rho \Delta x S \ddot h(x) = \sigma S (h'(x + \Delta x) - h'(x))
    \end{equation}
    \begin{equation}
        \ddot h(x) = \frac {\sigma} {\rho} \frac {h'(x + \Delta x) - h'(x)} {\Delta x}
    \end{equation}

    Перейдем к пределу при $\Delta x \longrightarrow 0$:
    \begin{equation}
        \frac{\partial^2 h}{\partial t^2} = \frac {\sigma} {\rho} \frac{\partial^2 h}{\partial x^2}
    \end{equation}

    Скорость распространения колебательного процесса внутри струны $u$:
    \begin{equation}
        \frac{\partial x^2}{\partial t^2} = \frac {\sigma} {\rho}
    \end{equation}
    \begin{equation}
        u = \frac{\partial x}{\partial t} = \sqrt {\frac {\sigma} {\rho}}
    \end{equation}

    \section{Колебание струны}

    Введем понятия погонной плотности $\rho_l$ и силы натяжения $F$:
    \begin{equation*}
        \rho_l = \rho S
    \end{equation*}

    \begin{equation*}
        F = \sigma S
    \end{equation*}

    Тогда:
    \begin{equation}
        u = \sqrt {\frac {F} {\rho_l}}
    \end{equation}

\end{document}
