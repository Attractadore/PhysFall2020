\documentclass{article}
\usepackage[english,russian]{babel}
\usepackage{amsmath}

\title{Определение скорости полета пули при помощи баллистического маятника}
\date{}
\author{Баргатин Михаил}

\begin{document}
    \maketitle
    \pagenumbering{gobble}
    \newpage
    \pagenumbering{arabic}

    \section{Метод баллистического маятника, совершающего поступательные движения}

    Вычислим скорость пули по отклонению баллистического маятника после попадания в него пули.
    \newline
    При попадании пули в маятник сохраняется импульс:
    \begin{equation}
        mv = (M + m)u
    \end{equation}
    где $m$ -- масса пули, $M$ -- масса маятника, $v$ -- скорость пули до попадания в маятник,
    $u$ -- скорость маятника после застревания в нем пули.
    \newline
    Масса маятника много больше массы пули:
    \begin{equation}
        M \gg m
    \end{equation}
    \begin{equation}
        mv \approx Mu
    \end{equation}
    \begin{equation}
        v = \frac{M}{m}u
    \end{equation}
    При подъеме маятника сохраняется энергия:
    \begin{equation}
        \frac{Mu^2}{2} = Mgh
    \end{equation}
    \begin{equation}
        u = \sqrt{2gh}
    \end{equation}
    где $h$ -- высота подъема маятника.
    \newline
    \begin{equation}
        h = l(1 - \cos \alpha) = l(1 - \sqrt{1-\sin^2\alpha})
    \end{equation}
    где $\alpha$ -- угол, на который откланяется маятник, $l$ -- длина нити, на которой подвешен маятник.
    \newline
    Горизонтальное отклонение маятника $\Delta x$:
    \begin{equation}
        \Delta x = l\sin\alpha
    \end{equation}
    Отклонение маятника мало:
    \begin{equation}
        \alpha \ll 1
    \end{equation}
    \begin{equation}
        h \approx l\left(\frac{\alpha^2}{2}\right)
    \end{equation}
    \begin{equation}
        \Delta x \approx l\alpha
    \end{equation}
    \begin{equation}
        \alpha = \frac{\Delta x}{l}
    \end{equation}
    \begin{equation}
        h = \frac{\Delta x^2}{2l}
    \end{equation}
    Из (6) и (13) получаем:
    \begin{equation}
        u = \sqrt{\frac{g \Delta x^2}{l}} = \Delta x \sqrt{\frac{g}{l}}
    \end{equation}
    Из (4) и (14) получаем:
    \begin{equation}
        v = \Delta x \frac{M}{m} \sqrt{\frac{g}{l}}
    \end{equation}

    \section{Метод крутильного баллистического маятника}

    Вычислим скорости пули по отклонение крутильного баллистического маятника после попадания в него пули.
    \newline
    При попадании пули в маятник его момент импульса $P$ становится равным:
    \begin{equation}
        P = I \omega = mvr
    \end{equation}
    \begin{equation}
        v = \frac{I \omega}{mr}
    \end{equation}
    где $m$ -- масса пули, $v$ -- скорость пули до попадания в маятник, $I$ -- момент инерции маятника,
    $\omega$ -- угловая скорость вращения маятника после попадания в него пули.
    \newline
    После попадания пули в маятник сохраняется энергия:
    \begin{equation}
        k \frac{\phi^2}{2} = I \frac{\omega^2}{2}
    \end{equation}
    где $k$ -- модуль кручения проволоки, маятника, $\phi$ -- максимальный угол, на который маятник откланяется.
    \begin{equation}
        \omega = \phi \sqrt{\frac{k}{I}}
    \end{equation}
    Из (17) и (19) получаем:
    \begin{equation}
        v = \phi \frac{\sqrt{kI}}{mr}
    \end{equation}

\end{document}
